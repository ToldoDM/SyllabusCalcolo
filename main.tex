\documentclass[12pt,a4paper,headings=optiontohead]{article}
\usepackage[utf8]{inputenc}
\usepackage[italian]{babel}
\usepackage[margin=1.8cm,bottom=7em]{geometry}
\usepackage[subpreambles=false]{standalone}
\usepackage{amsmath}
\usepackage{amssymb}
\usepackage{amsthm} 
\usepackage{cancel}
\usepackage{graphicx}
\usepackage{mathtools}
\usepackage{float}
\usepackage{enumitem}

\usepackage[normalem]{ulem}

\usepackage{blkarray}% http://ctan.org/pkg/blkarray

\newcommand{\matindex}[1]{\mbox{\scriptsize#1}}% Matrix index

\renewcommand{\qedsymbol}{\rule{0.7em}{0.7em}}
\DeclarePairedDelimiter{\abs}{\lvert}{\rvert}
\newcommand{\inter}{\begin{matrix}\prod\end{matrix}}
\newcommand{\verteq}{\rotatebox{90}{$\,=$}}
\newcommand{\equalto}[2]{\underset{\scriptstyle\overset{\mkern4mu\verteq}{#2}}{#1}}
\DeclarePairedDelimiter{\norma}{\lVert}{\rVert}
\newtheorem*{esempio}{Esempio}

\usepackage{import}
\usepackage{hyperref}
\begin{document}

%------------------------------------------------------------------------------------------------------
%------------------------------------------------------------------------------------------------------
%-----------------------------------------------INTESTAZIONE-------------------------------------------
%------------------------------------------------------------------------------------------------------
%------------------------------------------------------------------------------------------------------

\begin{titlepage}

\newcommand{\HRule}{\rule{\linewidth}{0.5mm}} % Defines a new command for the horizontal lines, change thickness here

\center % Center everything on the page
 
%----------------------------------------------------------------------------------------
%	HEADING SECTIONS
%----------------------------------------------------------------------------------------


\large Dalle dispense del\\[0.5cm] % Major heading such as course name
\textsc{\Large Prof. Marco Vianello}\\[1.5cm] % Minor heading such as course title
%----------------------------------------------------------------------------------------
%	TITLE SECTION
%----------------------------------------------------------------------------------------

\HRule \\[0.4cm]
{ \huge \bfseries Syllabus}\\
{ \huge \bfseries Dimostrazioni Irrinunciabili\\[0.15 cm]} % Title of your document
\HRule \\[1.5cm]
 
%----------------------------------------------------------------------------------------
%	AUTHOR SECTION
%----------------------------------------------------------------------------------------


%----------------------------------------------------------------------------------------
%	DATE SECTION
%----------------------------------------------------------------------------------------

\LARGE Università degli Studi di Padova\\[0.4cm] % Name of your university/college
\textsc{\large Dipartimento di Matematica}\\[0.05cm]
\textsc{\large Corso di Laurea in Informatica}\\[1cm] % Include a department/university logo - this will require the graphicx package

{\Large Anno accademico 2020 - 2021}\\[2cm] % Date, change the \today to a set date if you want to be precise

\vfill % Fill the rest of the page with whitespace

\end{titlepage}


\begin{center}
\pagebreak

\section*{Premessa}
\begin{minipage}{0.9\textwidth} \large

Questa raccolta di appunti non intende essere un sostituto allo studio completo degli argomenti di calcolo numerico. Gli appunti sono stati scritti secondo quanto studiato e capito, di conseguenza potrebbe contenere errori/non essere esaustivo nella risposta agli argomenti del syllabus.

\end{minipage}

\end{center}
\pagebreak

%------------------------------------------------------------------------------------------------------
%------------------------------------------------------------------------------------------------------
%-----------------------------------------------INDICE-------------------------------------------------
%------------------------------------------------------------------------------------------------------
%------------------------------------------------------------------------------------------------------

\tableofcontents

\newpage
\import{res/}{precis_macchina}
\newpage
\import{res/}{analisi_stab}
\import{res/}{conv_bisez}
\newpage
\import{res/}{stima_err_res_pesato}
\newpage
\import{res/}{conv_newton}

\end{document}
