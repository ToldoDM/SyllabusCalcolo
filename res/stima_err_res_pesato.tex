\section{Stima dell'errore con residuo pesato (metodo bisezione)}
Vogliamo stimare l'errore di bisezione, applicato nelle seguenti ipotesi:

\begin{equation*}
\begin{rcases}
f  \in C^1[a,b]\\
\{x_n\}  \in [c,d] \subseteq [a,b]\\
f'(x)\ne 0 \,,\, \forall x \in [c,d]
\end{rcases}
\Rightarrow e_n = |x_n-\xi| = \frac{|f(x_n)|}{|f'(z_n)|} \,,\,\text{   } n\ge n_0 \,,\,\text{   } z_n \in
\begin{cases}
(x_n, \xi) \\
(\xi, x_n)
\end{cases}
\end{equation*}
\begin{center}
	$C^1 \,indica \,derivabile \,1 \,volta\, con \,derivata\, continua.$
\end{center}
Dimostriamolo utilizzando il \underline{teorema del valor medio}
\[
\text{Sia } \,f\in C[a,b]\,\,\, \text{derivabile in }\, [a.b] \Rightarrow \exists z\in [a,b] : \frac{f(b)-f(a)}{b-a}=f'(z)
\]
Consideriamo il caso $\xi<x_n$ (se $x_n<\xi$ la dimostrazione è analoga)
\begin{equation*}
f(x_n)-f(\xi) = f'(z_n)(x_n-\xi), \text{ } z_n \in (\xi,x_n)
\end{equation*}
con $f(\xi)=0$, cioè
\begin{equation*}
|f(x_n)| = |f'(z_n)||x_n-\xi|
\end{equation*}
che si può riscrivere come
\begin{equation*}
e_n = |x_n-\xi| = \frac{|f(x_n)|}{|f'(z_n)|}
\end{equation*}
Osserviamo che:
\begin{itemize}
	\item $e_n$ è un "residuo pesato"
	\item $f'(x)\ne 0 \Rightarrow$ zero è semplice
	\item $e_n$ è una stima a posteriori (serve aver calcolato $x_n$)
\end{itemize}
Siccome non conosciamo $z_n$, diamo delle stime pratiche dell'errore:
\begin{itemize}
	\item Se è noto che $\abs*{f'(x)}\ge k>0 \Rightarrow e_n=\frac{|f(x_n)|}{|f'(z_n)|}\le  \frac{|f(x_n)|}{k}$
	\item Se $f'$ è nota, per $n$ abbastanza grande si ha
	\[
	\underbrace{f'(x_n)\approx f'(z_n)}_{\approx f'(\xi)} \Longrightarrow e_n \approx \frac{|f(x_n)|}{|f'(z_n)|}
	\]
	\item Se $f'$ non è nota, si può approssimare con
	\[
	f'(z_n)\approx \frac{f(x_n)-f(x_{n-1})}{x_n-x_{n-1}}\,,\,\,\, \text{per $n$ abbastanza grande}
	\]
\end{itemize}
